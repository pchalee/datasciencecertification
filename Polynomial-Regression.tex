\documentclass[]{article}
\usepackage{lmodern}
\usepackage{amssymb,amsmath}
\usepackage{ifxetex,ifluatex}
\usepackage{fixltx2e} % provides \textsubscript
\ifnum 0\ifxetex 1\fi\ifluatex 1\fi=0 % if pdftex
  \usepackage[T1]{fontenc}
  \usepackage[utf8]{inputenc}
\else % if luatex or xelatex
  \ifxetex
    \usepackage{mathspec}
  \else
    \usepackage{fontspec}
  \fi
  \defaultfontfeatures{Ligatures=TeX,Scale=MatchLowercase}
\fi
% use upquote if available, for straight quotes in verbatim environments
\IfFileExists{upquote.sty}{\usepackage{upquote}}{}
% use microtype if available
\IfFileExists{microtype.sty}{%
\usepackage{microtype}
\UseMicrotypeSet[protrusion]{basicmath} % disable protrusion for tt fonts
}{}
\usepackage[margin=1in]{geometry}
\usepackage{hyperref}
\hypersetup{unicode=true,
            pdftitle={Regression},
            pdfauthor={Veerasak Kritsanapraphan},
            pdfborder={0 0 0},
            breaklinks=true}
\urlstyle{same}  % don't use monospace font for urls
\usepackage{color}
\usepackage{fancyvrb}
\newcommand{\VerbBar}{|}
\newcommand{\VERB}{\Verb[commandchars=\\\{\}]}
\DefineVerbatimEnvironment{Highlighting}{Verbatim}{commandchars=\\\{\}}
% Add ',fontsize=\small' for more characters per line
\usepackage{framed}
\definecolor{shadecolor}{RGB}{248,248,248}
\newenvironment{Shaded}{\begin{snugshade}}{\end{snugshade}}
\newcommand{\AlertTok}[1]{\textcolor[rgb]{0.94,0.16,0.16}{#1}}
\newcommand{\AnnotationTok}[1]{\textcolor[rgb]{0.56,0.35,0.01}{\textbf{\textit{#1}}}}
\newcommand{\AttributeTok}[1]{\textcolor[rgb]{0.77,0.63,0.00}{#1}}
\newcommand{\BaseNTok}[1]{\textcolor[rgb]{0.00,0.00,0.81}{#1}}
\newcommand{\BuiltInTok}[1]{#1}
\newcommand{\CharTok}[1]{\textcolor[rgb]{0.31,0.60,0.02}{#1}}
\newcommand{\CommentTok}[1]{\textcolor[rgb]{0.56,0.35,0.01}{\textit{#1}}}
\newcommand{\CommentVarTok}[1]{\textcolor[rgb]{0.56,0.35,0.01}{\textbf{\textit{#1}}}}
\newcommand{\ConstantTok}[1]{\textcolor[rgb]{0.00,0.00,0.00}{#1}}
\newcommand{\ControlFlowTok}[1]{\textcolor[rgb]{0.13,0.29,0.53}{\textbf{#1}}}
\newcommand{\DataTypeTok}[1]{\textcolor[rgb]{0.13,0.29,0.53}{#1}}
\newcommand{\DecValTok}[1]{\textcolor[rgb]{0.00,0.00,0.81}{#1}}
\newcommand{\DocumentationTok}[1]{\textcolor[rgb]{0.56,0.35,0.01}{\textbf{\textit{#1}}}}
\newcommand{\ErrorTok}[1]{\textcolor[rgb]{0.64,0.00,0.00}{\textbf{#1}}}
\newcommand{\ExtensionTok}[1]{#1}
\newcommand{\FloatTok}[1]{\textcolor[rgb]{0.00,0.00,0.81}{#1}}
\newcommand{\FunctionTok}[1]{\textcolor[rgb]{0.00,0.00,0.00}{#1}}
\newcommand{\ImportTok}[1]{#1}
\newcommand{\InformationTok}[1]{\textcolor[rgb]{0.56,0.35,0.01}{\textbf{\textit{#1}}}}
\newcommand{\KeywordTok}[1]{\textcolor[rgb]{0.13,0.29,0.53}{\textbf{#1}}}
\newcommand{\NormalTok}[1]{#1}
\newcommand{\OperatorTok}[1]{\textcolor[rgb]{0.81,0.36,0.00}{\textbf{#1}}}
\newcommand{\OtherTok}[1]{\textcolor[rgb]{0.56,0.35,0.01}{#1}}
\newcommand{\PreprocessorTok}[1]{\textcolor[rgb]{0.56,0.35,0.01}{\textit{#1}}}
\newcommand{\RegionMarkerTok}[1]{#1}
\newcommand{\SpecialCharTok}[1]{\textcolor[rgb]{0.00,0.00,0.00}{#1}}
\newcommand{\SpecialStringTok}[1]{\textcolor[rgb]{0.31,0.60,0.02}{#1}}
\newcommand{\StringTok}[1]{\textcolor[rgb]{0.31,0.60,0.02}{#1}}
\newcommand{\VariableTok}[1]{\textcolor[rgb]{0.00,0.00,0.00}{#1}}
\newcommand{\VerbatimStringTok}[1]{\textcolor[rgb]{0.31,0.60,0.02}{#1}}
\newcommand{\WarningTok}[1]{\textcolor[rgb]{0.56,0.35,0.01}{\textbf{\textit{#1}}}}
\usepackage{graphicx,grffile}
\makeatletter
\def\maxwidth{\ifdim\Gin@nat@width>\linewidth\linewidth\else\Gin@nat@width\fi}
\def\maxheight{\ifdim\Gin@nat@height>\textheight\textheight\else\Gin@nat@height\fi}
\makeatother
% Scale images if necessary, so that they will not overflow the page
% margins by default, and it is still possible to overwrite the defaults
% using explicit options in \includegraphics[width, height, ...]{}
\setkeys{Gin}{width=\maxwidth,height=\maxheight,keepaspectratio}
\IfFileExists{parskip.sty}{%
\usepackage{parskip}
}{% else
\setlength{\parindent}{0pt}
\setlength{\parskip}{6pt plus 2pt minus 1pt}
}
\setlength{\emergencystretch}{3em}  % prevent overfull lines
\providecommand{\tightlist}{%
  \setlength{\itemsep}{0pt}\setlength{\parskip}{0pt}}
\setcounter{secnumdepth}{0}
% Redefines (sub)paragraphs to behave more like sections
\ifx\paragraph\undefined\else
\let\oldparagraph\paragraph
\renewcommand{\paragraph}[1]{\oldparagraph{#1}\mbox{}}
\fi
\ifx\subparagraph\undefined\else
\let\oldsubparagraph\subparagraph
\renewcommand{\subparagraph}[1]{\oldsubparagraph{#1}\mbox{}}
\fi

%%% Use protect on footnotes to avoid problems with footnotes in titles
\let\rmarkdownfootnote\footnote%
\def\footnote{\protect\rmarkdownfootnote}

%%% Change title format to be more compact
\usepackage{titling}

% Create subtitle command for use in maketitle
\providecommand{\subtitle}[1]{
  \posttitle{
    \begin{center}\large#1\end{center}
    }
}

\setlength{\droptitle}{-2em}

  \title{Regression}
    \pretitle{\vspace{\droptitle}\centering\huge}
  \posttitle{\par}
    \author{Veerasak Kritsanapraphan}
    \preauthor{\centering\large\emph}
  \postauthor{\par}
      \predate{\centering\large\emph}
  \postdate{\par}
    \date{1/17/2020}


\begin{document}
\maketitle

\hypertarget{loading-required-r-packages}{%
\subsubsection{Loading Required R
packages}\label{loading-required-r-packages}}

\begin{Shaded}
\begin{Highlighting}[]
\KeywordTok{library}\NormalTok{(tidyverse)}
\end{Highlighting}
\end{Shaded}

\begin{verbatim}
## ── Attaching packages ────────────────────── tidyverse 1.2.1 ──
\end{verbatim}

\begin{verbatim}
## ✓ ggplot2 3.2.1     ✓ purrr   0.3.3
## ✓ tibble  2.1.3     ✓ dplyr   0.8.3
## ✓ tidyr   1.0.0     ✓ stringr 1.4.0
## ✓ readr   1.3.1     ✓ forcats 0.4.0
\end{verbatim}

\begin{verbatim}
## ── Conflicts ───────────────────────── tidyverse_conflicts() ──
## x dplyr::filter() masks stats::filter()
## x dplyr::lag()    masks stats::lag()
\end{verbatim}

\begin{Shaded}
\begin{Highlighting}[]
\KeywordTok{library}\NormalTok{(caret)}
\end{Highlighting}
\end{Shaded}

\begin{verbatim}
## Loading required package: lattice
\end{verbatim}

\begin{verbatim}
## 
## Attaching package: 'caret'
\end{verbatim}

\begin{verbatim}
## The following object is masked from 'package:purrr':
## 
##     lift
\end{verbatim}

\begin{Shaded}
\begin{Highlighting}[]
\KeywordTok{theme_set}\NormalTok{(}\KeywordTok{theme_classic}\NormalTok{())}
\end{Highlighting}
\end{Shaded}

\hypertarget{preparing-the-data}{%
\subsubsection{Preparing the data}\label{preparing-the-data}}

We'll use the Boston data set {[}in MASS package{]}, for predicting the
median house value (mdev), in Boston Suburbs, based on the predictor
variable lstat (percentage of lower status of the population).

We'll randomly split the data into training set (80\% for building a
predictive model) and test set (20\% for evaluating the model).

\hypertarget{load-the-data}{%
\paragraph{Load the data}\label{load-the-data}}

\begin{Shaded}
\begin{Highlighting}[]
\KeywordTok{data}\NormalTok{(}\StringTok{"Boston"}\NormalTok{, }\DataTypeTok{package =} \StringTok{"MASS"}\NormalTok{)}

\CommentTok{# Split the data into training and test set}
\KeywordTok{set.seed}\NormalTok{(}\DecValTok{123}\NormalTok{)}
\NormalTok{training.samples <-}\StringTok{ }\NormalTok{Boston}\OperatorTok{$}\NormalTok{medv }\OperatorTok
\StringTok{  }\KeywordTok{createDataPartition}\NormalTok{(}\DataTypeTok{p =} \FloatTok{0.8}\NormalTok{, }\DataTypeTok{list =} \OtherTok{FALSE}\NormalTok{)}
\NormalTok{train.data  <-}\StringTok{ }\NormalTok{Boston[training.samples, ]}
\NormalTok{test.data <-}\StringTok{ }\NormalTok{Boston[}\OperatorTok{-}\NormalTok{training.samples, ]}
\end{Highlighting}
\end{Shaded}

First, visualize the scatter plot of the medv vs lstat variables as
follow:

\begin{Shaded}
\begin{Highlighting}[]
\KeywordTok{ggplot}\NormalTok{(train.data, }\KeywordTok{aes}\NormalTok{(lstat, medv) ) }\OperatorTok{+}
\StringTok{  }\KeywordTok{geom_point}\NormalTok{() }\OperatorTok{+}
\StringTok{  }\KeywordTok{stat_smooth}\NormalTok{()}
\end{Highlighting}
\end{Shaded}

\begin{verbatim}
## `geom_smooth()` using method = 'loess' and formula 'y ~ x'
\end{verbatim}

\includegraphics{Polynomial-Regression_files/figure-latex/unnamed-chunk-3-1.pdf}

The above scatter plot suggests a non-linear relationship between the
two variables

In the following sections, we start by computing linear and non-linear
regression models. Next, we'll compare the different models in order to
choose the best one for our data.

\hypertarget{linear-regression-linear-reg}{%
\subsection{Linear regression
\{linear-reg\}}\label{linear-regression-linear-reg}}

The standard linear regression model equation can be written as medv =
b0 + b1*lstat.

\hypertarget{compute-linear-regression-model}{%
\subsubsection{Compute linear regression
model:}\label{compute-linear-regression-model}}

\begin{Shaded}
\begin{Highlighting}[]
\CommentTok{# Build the model}
\NormalTok{model <-}\StringTok{ }\KeywordTok{lm}\NormalTok{(medv }\OperatorTok{~}\StringTok{ }\NormalTok{lstat, }\DataTypeTok{data =}\NormalTok{ train.data)}
\CommentTok{# Make predictions}
\NormalTok{predictions <-}\StringTok{ }\NormalTok{model }\OperatorTok\StringTok{ }\KeywordTok{predict}\NormalTok{(test.data)}
\CommentTok{# Model performance}
\KeywordTok{data.frame}\NormalTok{(}
  \DataTypeTok{RMSE =} \KeywordTok{RMSE}\NormalTok{(predictions, test.data}\OperatorTok{$}\NormalTok{medv),}
  \DataTypeTok{R2 =} \KeywordTok{R2}\NormalTok{(predictions, test.data}\OperatorTok{$}\NormalTok{medv)}
\NormalTok{)}
\end{Highlighting}
\end{Shaded}

\begin{verbatim}
##       RMSE        R2
## 1 6.073002 0.5354403
\end{verbatim}

\hypertarget{visualize-the-data}{%
\subsubsection{Visualize the data:}\label{visualize-the-data}}

\begin{Shaded}
\begin{Highlighting}[]
\KeywordTok{ggplot}\NormalTok{(train.data, }\KeywordTok{aes}\NormalTok{(lstat, medv) ) }\OperatorTok{+}
\StringTok{  }\KeywordTok{geom_point}\NormalTok{() }\OperatorTok{+}
\StringTok{  }\KeywordTok{stat_smooth}\NormalTok{(}\DataTypeTok{method =}\NormalTok{ lm, }\DataTypeTok{formula =}\NormalTok{ y }\OperatorTok{~}\StringTok{ }\NormalTok{x)}
\end{Highlighting}
\end{Shaded}

\includegraphics{Polynomial-Regression_files/figure-latex/unnamed-chunk-5-1.pdf}

\hypertarget{polynomial-regression}{%
\subsubsection{Polynomial regression}\label{polynomial-regression}}

The polynomial regression adds polynomial or quadratic terms to the
regression equation as follow:

medv=b0+b1∗lstat+b2∗lstat2

In R, to create a predictor x\^{}2 you should use the function I(), as
follow: I(x\^{}2). This raise x to the power 2.

The polynomial regression can be computed in R as follow:

\begin{Shaded}
\begin{Highlighting}[]
\KeywordTok{lm}\NormalTok{(medv }\OperatorTok{~}\StringTok{ }\NormalTok{lstat }\OperatorTok{+}\StringTok{ }\KeywordTok{I}\NormalTok{(lstat}\OperatorTok{^}\DecValTok{2}\NormalTok{), }\DataTypeTok{data =}\NormalTok{ train.data)}
\end{Highlighting}
\end{Shaded}

\begin{verbatim}
## 
## Call:
## lm(formula = medv ~ lstat + I(lstat^2), data = train.data)
## 
## Coefficients:
## (Intercept)        lstat   I(lstat^2)  
##    43.35115     -2.34018      0.04303
\end{verbatim}

An alternative simple solution is to use this:

\begin{Shaded}
\begin{Highlighting}[]
\KeywordTok{lm}\NormalTok{(medv }\OperatorTok{~}\StringTok{ }\KeywordTok{poly}\NormalTok{(lstat, }\DecValTok{2}\NormalTok{, }\DataTypeTok{raw =} \OtherTok{TRUE}\NormalTok{), }\DataTypeTok{data =}\NormalTok{ train.data)}
\end{Highlighting}
\end{Shaded}

\begin{verbatim}
## 
## Call:
## lm(formula = medv ~ poly(lstat, 2, raw = TRUE), data = train.data)
## 
## Coefficients:
##                 (Intercept)  poly(lstat, 2, raw = TRUE)1  
##                    43.35115                     -2.34018  
## poly(lstat, 2, raw = TRUE)2  
##                     0.04303
\end{verbatim}

The following example computes a sixfth-order polynomial fit:

\begin{Shaded}
\begin{Highlighting}[]
\KeywordTok{lm}\NormalTok{(medv }\OperatorTok{~}\StringTok{ }\KeywordTok{poly}\NormalTok{(lstat, }\DecValTok{6}\NormalTok{, }\DataTypeTok{raw =} \OtherTok{TRUE}\NormalTok{), }\DataTypeTok{data =}\NormalTok{ train.data) }\OperatorTok
\StringTok{  }\KeywordTok{summary}\NormalTok{()}
\end{Highlighting}
\end{Shaded}

\begin{verbatim}
## 
## Call:
## lm(formula = medv ~ poly(lstat, 6, raw = TRUE), data = train.data)
## 
## Residuals:
##      Min       1Q   Median       3Q      Max 
## -14.2327  -3.2366  -0.7399   2.0233  26.4959 
## 
## Coefficients:
##                               Estimate Std. Error t value Pr(>|t|)    
## (Intercept)                  7.137e+01  5.999e+00  11.897  < 2e-16 ***
## poly(lstat, 6, raw = TRUE)1 -1.445e+01  3.224e+00  -4.483 9.65e-06 ***
## poly(lstat, 6, raw = TRUE)2  1.868e+00  6.263e-01   2.982  0.00304 ** 
## poly(lstat, 6, raw = TRUE)3 -1.316e-01  5.731e-02  -2.296  0.02217 *  
## poly(lstat, 6, raw = TRUE)4  4.984e-03  2.659e-03   1.874  0.06161 .  
## poly(lstat, 6, raw = TRUE)5 -9.558e-05  6.034e-05  -1.584  0.11395    
## poly(lstat, 6, raw = TRUE)6  7.291e-07  5.300e-07   1.376  0.16970    
## ---
## Signif. codes:  0 '***' 0.001 '**' 0.01 '*' 0.05 '.' 0.1 ' ' 1
## 
## Residual standard error: 5.284 on 400 degrees of freedom
## Multiple R-squared:  0.6836, Adjusted R-squared:  0.6789 
## F-statistic:   144 on 6 and 400 DF,  p-value: < 2.2e-16
\end{verbatim}

From the output above, it can be seen that polynomial terms beyond the
fith order are not significant. So, just create a fifth polynomial
regression model as follow:

\hypertarget{build-the-fifth-polynomial-model}{%
\subsubsection{Build the fifth polynomial
model}\label{build-the-fifth-polynomial-model}}

\begin{Shaded}
\begin{Highlighting}[]
\NormalTok{model <-}\StringTok{ }\KeywordTok{lm}\NormalTok{(medv }\OperatorTok{~}\StringTok{ }\KeywordTok{poly}\NormalTok{(lstat, }\DecValTok{5}\NormalTok{, }\DataTypeTok{raw =} \OtherTok{TRUE}\NormalTok{), }\DataTypeTok{data =}\NormalTok{ train.data)}
\end{Highlighting}
\end{Shaded}

\hypertarget{make-predictions}{%
\subsubsection{Make predictions}\label{make-predictions}}

\begin{Shaded}
\begin{Highlighting}[]
\NormalTok{predictions <-}\StringTok{ }\NormalTok{model }\OperatorTok\StringTok{ }\KeywordTok{predict}\NormalTok{(test.data)}
\end{Highlighting}
\end{Shaded}

\hypertarget{model-performance}{%
\subsubsection{Model performance}\label{model-performance}}

\begin{Shaded}
\begin{Highlighting}[]
\KeywordTok{data.frame}\NormalTok{(}
  \DataTypeTok{RMSE =} \KeywordTok{RMSE}\NormalTok{(predictions, test.data}\OperatorTok{$}\NormalTok{medv),}
  \DataTypeTok{R2 =} \KeywordTok{R2}\NormalTok{(predictions, test.data}\OperatorTok{$}\NormalTok{medv)}
\NormalTok{)}
\end{Highlighting}
\end{Shaded}

\begin{verbatim}
##      RMSE        R2
## 1 4.96127 0.6887199
\end{verbatim}

Visualize the fith polynomial regression line as follow:

\begin{Shaded}
\begin{Highlighting}[]
\KeywordTok{ggplot}\NormalTok{(train.data, }\KeywordTok{aes}\NormalTok{(lstat, medv) ) }\OperatorTok{+}
\StringTok{  }\KeywordTok{geom_point}\NormalTok{() }\OperatorTok{+}
\StringTok{  }\KeywordTok{stat_smooth}\NormalTok{(}\DataTypeTok{method =}\NormalTok{ lm, }\DataTypeTok{formula =}\NormalTok{ y }\OperatorTok{~}\StringTok{ }\KeywordTok{poly}\NormalTok{(x, }\DecValTok{5}\NormalTok{, }\DataTypeTok{raw =} \OtherTok{TRUE}\NormalTok{))}
\end{Highlighting}
\end{Shaded}

\includegraphics{Polynomial-Regression_files/figure-latex/unnamed-chunk-12-1.pdf}

\hypertarget{log-transformation}{%
\subsubsection{Log transformation}\label{log-transformation}}

When you have a non-linear relationship, you can also try a logarithm
transformation of the predictor variables:

\hypertarget{build-the-model}{%
\subsubsection{Build the model}\label{build-the-model}}

\begin{Shaded}
\begin{Highlighting}[]
\NormalTok{model <-}\StringTok{ }\KeywordTok{lm}\NormalTok{(medv }\OperatorTok{~}\StringTok{ }\KeywordTok{log}\NormalTok{(lstat), }\DataTypeTok{data =}\NormalTok{ train.data)}
\end{Highlighting}
\end{Shaded}

\hypertarget{make-predictions-1}{%
\subsubsection{Make predictions}\label{make-predictions-1}}

\begin{Shaded}
\begin{Highlighting}[]
\NormalTok{predictions <-}\StringTok{ }\NormalTok{model }\OperatorTok\StringTok{ }\KeywordTok{predict}\NormalTok{(test.data)}
\end{Highlighting}
\end{Shaded}

\hypertarget{model-performance-1}{%
\subsubsection{Model performance}\label{model-performance-1}}

\begin{Shaded}
\begin{Highlighting}[]
\KeywordTok{data.frame}\NormalTok{(}
  \DataTypeTok{RMSE =} \KeywordTok{RMSE}\NormalTok{(predictions, test.data}\OperatorTok{$}\NormalTok{medv),}
  \DataTypeTok{R2 =} \KeywordTok{R2}\NormalTok{(predictions, test.data}\OperatorTok{$}\NormalTok{medv)}
\NormalTok{)}
\end{Highlighting}
\end{Shaded}

\begin{verbatim}
##       RMSE        R2
## 1 5.243127 0.6565156
\end{verbatim}

Visualize the data:

\begin{Shaded}
\begin{Highlighting}[]
\KeywordTok{ggplot}\NormalTok{(train.data, }\KeywordTok{aes}\NormalTok{(lstat, medv) ) }\OperatorTok{+}
\StringTok{  }\KeywordTok{geom_point}\NormalTok{() }\OperatorTok{+}
\StringTok{  }\KeywordTok{stat_smooth}\NormalTok{(}\DataTypeTok{method =}\NormalTok{ lm, }\DataTypeTok{formula =}\NormalTok{ y }\OperatorTok{~}\StringTok{ }\KeywordTok{log}\NormalTok{(x))}
\end{Highlighting}
\end{Shaded}

\includegraphics{Polynomial-Regression_files/figure-latex/unnamed-chunk-16-1.pdf}

\hypertarget{spline-regression}{%
\subsubsection{Spline regression}\label{spline-regression}}

Polynomial regression only captures a certain amount of curvature in a
nonlinear relationship. An alternative, and often superior, approach to
modeling nonlinear relationships is to use splines.

Splines provide a way to smoothly interpolate between fixed points,
called knots. Polynomial regression is computed between knots. In other
words, splines are series of polynomial segments strung together,
joining at knots.

The R package splines includes the function bs for creating a b-spline
term in a regression model.

You need to specify two parameters: the degree of the polynomial and the
location of the knots. We'll place the knots at the lower quartile, the
median quartile, and the upper quartile:

\begin{Shaded}
\begin{Highlighting}[]
\NormalTok{knots <-}\StringTok{ }\KeywordTok{quantile}\NormalTok{(train.data}\OperatorTok{$}\NormalTok{lstat, }\DataTypeTok{p =} \KeywordTok{c}\NormalTok{(}\FloatTok{0.25}\NormalTok{, }\FloatTok{0.5}\NormalTok{, }\FloatTok{0.75}\NormalTok{))}
\end{Highlighting}
\end{Shaded}

We'll create a model using a cubic spline (degree = 3):

\begin{Shaded}
\begin{Highlighting}[]
\KeywordTok{library}\NormalTok{(splines)}
\CommentTok{# Build the model}
\NormalTok{knots <-}\StringTok{ }\KeywordTok{quantile}\NormalTok{(train.data}\OperatorTok{$}\NormalTok{lstat, }\DataTypeTok{p =} \KeywordTok{c}\NormalTok{(}\FloatTok{0.25}\NormalTok{, }\FloatTok{0.5}\NormalTok{, }\FloatTok{0.75}\NormalTok{))}
\NormalTok{model <-}\StringTok{ }\KeywordTok{lm}\NormalTok{ (medv }\OperatorTok{~}\StringTok{ }\KeywordTok{bs}\NormalTok{(lstat, }\DataTypeTok{knots =}\NormalTok{ knots), }\DataTypeTok{data =}\NormalTok{ train.data)}
\CommentTok{# Make predictions}
\NormalTok{predictions <-}\StringTok{ }\NormalTok{model }\OperatorTok\StringTok{ }\KeywordTok{predict}\NormalTok{(test.data)}
\CommentTok{# Model performance}
\KeywordTok{data.frame}\NormalTok{(}
  \DataTypeTok{RMSE =} \KeywordTok{RMSE}\NormalTok{(predictions, test.data}\OperatorTok{$}\NormalTok{medv),}
  \DataTypeTok{R2 =} \KeywordTok{R2}\NormalTok{(predictions, test.data}\OperatorTok{$}\NormalTok{medv)}
\NormalTok{)}
\end{Highlighting}
\end{Shaded}

\begin{verbatim}
##       RMSE        R2
## 1 4.970949 0.6881139
\end{verbatim}

Note that, the coefficients for a spline term are not interpretable.

Visualize the cubic spline as follow:

\begin{Shaded}
\begin{Highlighting}[]
\KeywordTok{ggplot}\NormalTok{(train.data, }\KeywordTok{aes}\NormalTok{(lstat, medv) ) }\OperatorTok{+}
\StringTok{  }\KeywordTok{geom_point}\NormalTok{() }\OperatorTok{+}
\StringTok{  }\KeywordTok{stat_smooth}\NormalTok{(}\DataTypeTok{method =}\NormalTok{ lm, }\DataTypeTok{formula =}\NormalTok{ y }\OperatorTok{~}\StringTok{ }\NormalTok{splines}\OperatorTok{::}\KeywordTok{bs}\NormalTok{(x, }\DataTypeTok{df =} \DecValTok{3}\NormalTok{))}
\end{Highlighting}
\end{Shaded}

\includegraphics{Polynomial-Regression_files/figure-latex/unnamed-chunk-19-1.pdf}

Generalized additive models Once you have detected a non-linear
relationship in your data, the polynomial terms may not be flexible
enough to capture the relationship, and spline terms require specifying
the knots.

Generalized additive models, or GAM, are a technique to automatically
fit a spline regression. This can be done using the mgcv R package:

\begin{Shaded}
\begin{Highlighting}[]
\KeywordTok{library}\NormalTok{(mgcv)}
\end{Highlighting}
\end{Shaded}

\begin{verbatim}
## Loading required package: nlme
\end{verbatim}

\begin{verbatim}
## 
## Attaching package: 'nlme'
\end{verbatim}

\begin{verbatim}
## The following object is masked from 'package:dplyr':
## 
##     collapse
\end{verbatim}

\begin{verbatim}
## This is mgcv 1.8-28. For overview type 'help("mgcv-package")'.
\end{verbatim}

\begin{Shaded}
\begin{Highlighting}[]
\CommentTok{# Build the model}
\NormalTok{model <-}\StringTok{ }\KeywordTok{gam}\NormalTok{(medv }\OperatorTok{~}\StringTok{ }\KeywordTok{s}\NormalTok{(lstat), }\DataTypeTok{data =}\NormalTok{ train.data)}
\CommentTok{# Make predictions}
\NormalTok{predictions <-}\StringTok{ }\NormalTok{model }\OperatorTok\StringTok{ }\KeywordTok{predict}\NormalTok{(test.data)}
\CommentTok{# Model performance}
\KeywordTok{data.frame}\NormalTok{(}
  \DataTypeTok{RMSE =} \KeywordTok{RMSE}\NormalTok{(predictions, test.data}\OperatorTok{$}\NormalTok{medv),}
  \DataTypeTok{R2 =} \KeywordTok{R2}\NormalTok{(predictions, test.data}\OperatorTok{$}\NormalTok{medv)}
\NormalTok{)}
\end{Highlighting}
\end{Shaded}

\begin{verbatim}
##      RMSE        R2
## 1 5.01729 0.6838236
\end{verbatim}

The term s(lstat) tells the gam() function to find the ``best'' knots
for a spline term.

\hypertarget{visualize-the-data-1}{%
\subsubsection{Visualize the data:}\label{visualize-the-data-1}}

\begin{Shaded}
\begin{Highlighting}[]
\KeywordTok{ggplot}\NormalTok{(train.data, }\KeywordTok{aes}\NormalTok{(lstat, medv) ) }\OperatorTok{+}
\StringTok{  }\KeywordTok{geom_point}\NormalTok{() }\OperatorTok{+}
\StringTok{  }\KeywordTok{stat_smooth}\NormalTok{(}\DataTypeTok{method =}\NormalTok{ gam, }\DataTypeTok{formula =}\NormalTok{ y }\OperatorTok{~}\StringTok{ }\KeywordTok{s}\NormalTok{(x))}
\end{Highlighting}
\end{Shaded}

\includegraphics{Polynomial-Regression_files/figure-latex/unnamed-chunk-21-1.pdf}

\hypertarget{comparing-the-models}{%
\subsubsection{Comparing the models}\label{comparing-the-models}}

From analyzing the RMSE and the R2 metrics of the different models, it
can be seen that the polynomial regression, the spline regression and
the generalized additive models outperform the linear regression model
and the log transformation approaches.


\end{document}
